\chapter{Introduction} \label{chap:chap-1}

% if you want a short header you can use the following command
% \chapter[short-header-name]{chapter-title} \label{chap:chap-1}


%% add your chapter text here

In a sense, Democritus can be regarded as the intellectual father of what we now recognize as ``particle physics.'' The ancient Greek atomists, including Democritus and his mentor Leucippus, proposed that all substances observed in the natural world could be reduced to a set of indivisible, fundamental constituents they termed ``atoms.'' According to their philosophical framework, these atoms were eternal, indestructible, and varied only in shape, size, and motion, thereby giving rise to the macroscopic diversity of materials. This notion, while purely speculative, laid the groundwork for later scientific developments concerning the composition of matter.

By the 19th century, the concept of atoms had been refined and formalized within the framework of chemistry, where the term ``atom'' came to denote the smallest unit of a chemical element that retains its identity in a reaction. However, the atomic theory of chemistry, as it developed through the work of figures like John Dalton, Dmitri Mendeleev, and later Niels Bohr, eventually revealed that these so-called atoms were not, in fact, indivisible. Advances in experimental physics, particularly the discovery of the electron by J.J. Thomson in 1897 and the atomic nucleus by Ernest Rutherford in 1911, demonstrated that what chemists referred to as ``atoms'' were themselves composite structures, composed of subatomic particles—electrons, protons, and neutrons. Thus, the original atomist vision of fundamental, structureless building blocks found a more fitting realization not at the atomic scale, but in what modern physics refers to as elementary particles.

To contextualize the significance of the analysis presented in this thesis, it is essential to first examine the Standard Model of particle physics. The Standard Model represents the most comprehensive theoretical framework for classifying and describing the fundamental particles and their interactions, as currently understood. Developed over the course of the mid-20th century, with its foundations laid in quantum field theory and gauge symmetries, the Standard Model reached its modern formulation in the 1970s. It successfully unifies three of the four fundamental forces---the electromagnetic force, the weak nuclear force, and the strong nuclear force---within a single theoretical structure, while classifying all known elementary particles into two broad categories: fermions, which constitute matter, and bosons, which mediate interactions.

Despite its tremendous success, both in terms of predictive power and its alignment with experimental data, the Standard Model is known to be incomplete. While it provides an extraordinarily accurate description of particle interactions at accessible energy scales, it does not incorporate gravity, nor does it account for dark matter, dark energy, or the observed mass hierarchy of elementary particles. Nevertheless, it remains the most rigorously tested and empirically validated theory of fundamental particles to date. The discovery of the Higgs boson at the Large Hadron Collider in 2012 provided the final experimental confirmation of the Standard Model’s mechanism for mass generation of elementary particles, reinforcing its status as a cornerstone of modern physics. Yet, ongoing research continues to probe the limitations of this framework, seeking insights that may lead to a deeper, more fundamental theory that extends beyond the Standard Model.
% ~\cite{TheStand35}.








% In a sense, Democritus is the father of what we now call ``particle physics.''
% The ancient Greek atomists, including Democritus, posited that all manner of substances found in the world around us could be constructed from the same fundamental building blocks termed ``atoms.''

% Come the 19th century, the term ``atom'' was adopted into the terminology of chemists, such that in modern English, ``atoms'' refer to the building blocks of the chemical elements. 
% But these “atoms” at the level of chemical elements aren’t really fundamental. That is to say that they are in turn made of smaller constituent parts.
% What Democritus and his fellow atomists were philosophizing about were what we would now call the elementary particles.

% To understand the value of the analysis and additional work presented in this thesis, we must first take a quick look at the Standard Model of particle physics. The Standard Model is currently our best attempt at classifying the known particles, at the level of what we can call ``elementary building blocks.'' The Standard Model in its current form has been around for a few decades (since the 1970's), and has proven to be a reliable way to make predictions of physical phenomena, as well as explain the vast majority of experimental observations \cite{TheStand35}.

